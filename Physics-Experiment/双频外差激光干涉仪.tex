\documentclass{article}

% Chinese Support using xeCJK
% \usepackage{xeCJK}
% \setCJKmainfont{SimSun}

% Chinese Support using CTeX
\usepackage{ctex}

% Math Support
\usepackage{amsmath}
\usepackage{amsfonts}
\usepackage{amssymb}
\usepackage{wasysym}
\newcommand{\angstrom}{\text{\normalfont\AA}}
\usepackage{fancyhdr}

% Graphics Support
\usepackage{graphicx}
\usepackage{float}

% Reduced page margin
\usepackage{geometry}
\geometry{a4paper,scale=0.8}

\usepackage{caption}
\usepackage{subcaption}

% d and e should be math operators
\newcommand*{\dif}{\mathop{}\!\mathrm{d}}
\newcommand*{\md}{\mathop{}\!\mathrm{d}}
\newcommand*{\me}{\mathrm{e}}

% No indent for each paragraph
% \usepackage{parskip}
% \setlength{\parindent}{0cm}

% Bold style for Greek letters
\usepackage{bm}
\let\Oldmathbf\mathbf
\renewcommand{\mathbf}[1]{\boldsymbol{\Oldmathbf{#1}}}

% More space for dfrac in cell
\usepackage{cellspace}
\setlength{\cellspacetoplimit}{5pt}
\setlength{\cellspacebottomlimit}{5pt}

% SI units
\newcommand{\si}[1]{\  \mathrm{#1}}

% Multi-line author information
\usepackage{authblk}
\author{物理(4+4)1801 \quad  胡喜平 \quad U201811966}
\affil{个人网站 https://hxp.plus/ \quad 电子邮件 hxp201406@gmail.com}

\title{近代物理实验预习笔记——双频外差激光干涉仪}

\pagestyle{fancy}
\fancyhf{}
\lhead{源码地址:https://github.com/hxp-plus/Notes/tree/master/Physics-Experiment}
\rfoot{第 \thepage 页}
\renewcommand{\headrulewidth}{1pt}
\renewcommand{\footrulewidth}{1pt}

\begin{document}

\maketitle\thispagestyle{fancy}

\section{实验内容}

\begin{itemize}
\item 使用声光调制器(AOM)对激光光束进行调制,产生不同频率的激光。并搭建干涉仪光路。
\item 不考虑偏振的情况下,观察和比较参考光和测量光的干涉信号,通过两者相位差测量决定光程差,得出相位差与反射镜移动位移的函数关系。
\end{itemize}

\section{实验原理和注意事项}

非偏振双频激光干涉仪如图所示,其中两束氦氖激光存在无论是到PD1还是PD2都存在一定的光程差,为了方便讨论将图上四个位置用字母A、B、C、D表示。PD表示光电测量器,AOM表示声光调制器。

\begin{figure}[H]
  \centering
  \includegraphics[width=0.9\linewidth]{figures/非偏振双频激光干涉仪}
  \caption{非偏振双频激光干涉仪示意图}
\end{figure}

其中抵达PD1的是参考光,抵达PD2的是测量光。实验中直接测量的量是两组干涉激光的测量信号相位差,间接测量反射镜M2与M1的相对位移。

我们假设刚开始两个反射镜水平方向是没有位移的,那么对于\textbf{参考光},频率为$f_2$的光比频率为$f_1$的光多走的距离是$2 \overline{AB}$。而对于\textbf{测量光},频率为$f_2$的光比频率为$f_1$的光多走的距离是$2 \overline{BC}$。

因此为了防止出现奇怪的情况,在搭建实验光路的时候,应当使得\textbf{矩形ABCD}是\textbf{正方形}。

\end{document} 
