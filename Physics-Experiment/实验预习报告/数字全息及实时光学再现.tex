\documentclass{article}

% Chinese Support using xeCJK
% \usepackage{xeCJK}
% \setCJKmainfont{SimSun}

% Chinese Support using CTeX
\usepackage{ctex}

% Math Support
\usepackage{amsmath}
\usepackage{amsfonts}
\usepackage{amssymb}
\usepackage{wasysym}
\newcommand{\angstrom}{\text{\normalfont\AA}}
\usepackage{fancyhdr}

% Graphics Support
\usepackage{graphicx}
\usepackage{float}
\restylefloat{table}

% Reduced page margin
\usepackage{geometry}
\geometry{a4paper,scale=0.8}

\usepackage{caption}
\usepackage{subcaption}

% d and e should be math operators
\newcommand*{\dif}{\mathop{}\!\mathrm{d}}
\newcommand*{\md}{\mathop{}\!\mathrm{d}}
\newcommand*{\me}{\mathrm{e}}

% No indent for each paragraph
% \usepackage{parskip}
% \setlength{\parindent}{0cm}

% Bold style for Greek letters
\usepackage{bm}
\let\Oldmathbf\mathbf
\renewcommand{\mathbf}[1]{\boldsymbol{\Oldmathbf{#1}}}

% More space for dfrac in cell
\usepackage{cellspace}
\setlength{\cellspacetoplimit}{5pt}
\setlength{\cellspacebottomlimit}{5pt}

% SI units
\newcommand{\si}[1]{\  \mathrm{#1}}

% Multi-line author information
\usepackage{authblk}
\author{物理(4+4)1801 \quad  胡喜平 \quad U201811966}
\affil{个人网站 https://hxp.plus/ \quad 电子邮件 hxp201406@gmail.com}

\title{综合物理实验预习笔记——数字全息及实时光学再现}

\pagestyle{fancy}
\fancyhf{}
\lhead{源码地址:https://github.com/hxp-plus/Notes/tree/master/Physics-Experiment}
\rfoot{第 \thepage 页}
\renewcommand{\headrulewidth}{1pt}
\renewcommand{\footrulewidth}{1pt}


\begin{document}

\maketitle\thispagestyle{fancy}




\section{实验原理}

\subsection{全息图的记录}

由惠更斯-菲涅尔原理可知,任何物体发出的光波可以看成是由物体上各点发出的球面波总和。光的电场分布可以由复振幅表示为

\begin{equation}
  \vec{E}_o \left( \vec{r} \right) = A_o \left( \vec{r} \right)  \exp \left[ - i \varphi_o \left( r \right) \right]
\end{equation}
但是传统相机都只能记录光强不能记录相位,因此相片没有立体感。

要想记录光的全部信息,需要借助全息图。常用的方法是干涉法,使具有振幅和相位信息的物体发出的光,和一束参考光干涉,参考光的振幅相位已知,测量到的干涉光的相位振幅已知,因此能反推出物体发出的光的相位和振幅。

假设物光$\vec{E}_o$和参考光$\vec{E}_r$照射在一个感光材料上,感光材料可以记录光强,其中

\begin{equation}
  \label{eq:物光}
  \vec{E}_o \left( x, y \right) = A_o \left( x, y \right) \exp \left[ - i \varphi_o \left( x, y \right) \right]
\end{equation} 

\begin{equation}
  \label{eq:参考光}
  \vec{E}_r \left( x, y \right) = A_r \left( x, y \right) \exp \left[ - i \varphi_r \left( x, y \right) \right]
\end{equation}

这两束光叠加后产生的光强是

\begin{equation}
  I = \left( \vec{E}_o + \vec{E}_r \right) \left( \vec{E}_o + \vec{E}_r \right)^{*} = E_o^2 + E_r^2 + \vec{E}_o \vec{E}_r^{*} + \vec{E}_o^{*} \vec{E}_r
\end{equation}

代入\ref{eq:物光}和\ref{eq:参考光},得出

\begin{equation}
  \label{eq:全息图记录}
  I \left( x, y \right) = A_o^2 + A_r^2 + A_o A_r \left\{ \exp \left[ i \left( \varphi_r - \varphi_o \right) \right] + \exp \left[ i \left( \varphi_o - \varphi_r \right) \right] \right\}
\end{equation}

可以看出,参考光的引入,使得物光的相位信息被转化成了光强,因此可以被感光的芯片或者电子元器件捕获。

将记录介质曝光即可得到全息图,其中全息图上每一点的振幅透射系数和曝光时的光强成线性关系,
即

\begin{equation}
  \label{eq:线性关系}
  t \left( x,y \right) = \beta_0 + \beta I \left( x, y \right)
\end{equation}

\subsection{全息图的再现}

将感光介质生成的全息图放在透射式空间光调制器上,另照射到透射式空间光调制器的再现光为

\begin{equation}
  \label{eq:再现光}
  \vec{E}_r' \left( x,y \right) = A_r' \exp \left( - i \varphi_r' \right)
\end{equation}

则经过空间调制器后光的复振幅为

\begin{equation}
  \label{eq:投射复振幅}
  \vec{E}_g \left( x,y \right) = \vec{E}_r' t
\end{equation}

将\ref{eq:全息图记录}、\ref{eq:再现光}和\ref{eq:线性关系}代入,得到

\begin{equation}
  \label{eq:全息再现}
  \vec{E}_g = A_r' \exp \left( - i \varphi_r' \right) \left[ \beta_0 + \beta \left( A_o^2 + A_r^2 + A_o A_r \left\{ \exp \left[ i \left( \varphi_r - \varphi_o \right) \right] + \exp \left[ i \left( \varphi_o - \varphi_r \right) \right] \right\} \right) \right]
\end{equation}

当复现光和参考光相位相同,即$\varphi_r = \varphi_r'$时

\begin{equation}
  \label{eq:复现光参考光相等}
  \vec{E}_g \left( x,y \right) = \left( \beta_0 + \beta A_o^2 + \beta A_r^2 \right) \vec{E}_r'
  + \beta A_r A_r' \vec{E}_o + \beta A_r A_r' \vec{E}_o^{*} \exp \left[ - 2 i \varphi_r \right]
\end{equation}

第一项是投射过来的复现光,算噪声。第二项是沿着原来物光传播方向传播的投射光,呈原来物体的虚像。第三项是物光的共轭光波,在原来物体对称位置呈实像。

\section{实验内容}

\begin{itemize}
\item 计算机模拟全息(数字记录,数字再现)
\item 可视数字全息(数字记录,光学再现)
\item 数字全息(光学记录,数字再现)
\item 实时传统全息(光学记录,光学再现)
\end{itemize}

\end{document} 

%%% Local Variables:
%%% mode: latex
%%% TeX-master: t
%%% End:
