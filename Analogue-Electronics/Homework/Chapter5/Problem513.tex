\documentclass{article}
% Chinese
% \documentclass[UTF8, nofonts, mathptmx, 12pt, onecolumn]{article}
% \usepackage{xeCJK}
% \setCJKmainfont{SimSun}
\usepackage{amsmath}
\usepackage{amsfonts}
\usepackage{amssymb}
\usepackage{wasysym}
% \usepackage{ctex}
\usepackage{graphicx}
\usepackage{float}
\usepackage{geometry}
\geometry{a4paper,scale=0.8}
\usepackage{caption}
\usepackage{subcaption}
% \newcommand{\oiint}{\mathop{{\int\!\!\!\!\!\int}\mkern-21mu \bigcirc} {}}
\newcommand*{\dif}{\mathop{}\!\mathrm{d}}
\newcommand*{\md}{\mathop{}\!\mathrm{d}}
\newcommand*{\me}{\mathrm{e}}

% \usepackage{parskip}
% \setlength{\parindent}{0cm}

\usepackage{bm}
\let\Oldmathbf\mathbf
\renewcommand{\mathbf}[1]{\boldsymbol{\Oldmathbf{#1}}}
\let\eqnarray\align

\author{Xiping Hu}
\usepackage{authblk}
\author{Xiping Hu}
\affil{https://hxp.plus/}
\title{Homework for Chapter 5}

\begin{document}
\maketitle

\begin{figure}[H]
  \centering
  \includegraphics[width=\linewidth]{figures/Problem513}
  \label{fig:}
\end{figure}

\paragraph{Problem 1}

\begin{equation*}
  \left\{
    \begin{aligned}
      & I_C V_{CE} < P_{CM} \\
      & I_C < I_{CM} 
    \end{aligned}
  \right.
  \Rightarrow
  \left\{
    \begin{aligned}
      & I_C < 15 \  \mathrm{mA} \\
      & I_C < 100 \  \mathrm{mA}
    \end{aligned}
  \right.
\Rightarrow
  \begin{aligned}
    I_{C} < 15 \  \mathrm{mA}
  \end{aligned}
\end{equation*}

\paragraph{Problem 2}

\begin{equation*}
  \left\{
  \begin{aligned}
    & I_C V_{CE} < P_{CM} \\
    & V_{CE} < V_{\left( BR \right)CEO} 
  \end{aligned}
  \right.
\Rightarrow
  \left\{
  \begin{aligned}
    & V_{CE} < \dfrac{P_{CM}}{I_C} =  150 \  \mathrm{V} \\
    & V_{CE} < 30 \  \mathrm{V}
  \end{aligned}
  \right.
\Rightarrow
  \begin{aligned}
    V_{CE} < 30 \  \mathrm{V}
  \end{aligned}
\end{equation*}




\end{document}