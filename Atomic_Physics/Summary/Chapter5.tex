\chapter{Atoms with Two Electrons}

\section{LS Coupling}

\begin{table*}[h]
  \centering
  \begin{tabular}{|Sc|Sc|Sc|}
    \hline
    Compound Self-spin Angular Momentum & $S = 1,0$ & $P_S = \sqrt{S \left( S + 1 \right)} \hbar$  \\
    \hline
    Compound Orbital Angular Momentum & $L = | l_1 + l_2 |, | l_1 + l_2 | - 1, \dots ,|l_1 - l_2|$ & $P_S = \sqrt{L \left( L + 1 \right)} \hbar$  \\
    \hline
    Compound Total Angular Momentum & $J = | L + S |, | L + S | - 1, \dots ,|L - S |$ & $P_S = \sqrt{S \left( S + 1 \right)} \hbar$  \\
    \hline
  \end{tabular}
\end{table*}

\subsection{Hund's Law}

\begin{itemize}
\item For a given electron configuration, the term with maximum multiplicity has the lowest energy. The multiplicity is equal to $2S+1$ , where $S$ is the total spin angular momentum for all electrons.
\item For a given multiplicity, the term with the largest value of the total orbital angular momentum quantum number  $L$ ,  has the lowest energy.

\item For a given term, in an atom with outermost subshell half-filled or less, the level with the lowest value of the total angular momentum quantum number $J$ , (for the operator $J = L + S$) lies lowest in energy. If the outermost shell is more than half-filled, the level with the highest value of $J$, is lowest in energy.
\end{itemize}

\subsection{Rules of Transition}

\begin{table*}[h]
  \centering
  \begin{tabular}{|Sc|Sc|Sc|}
    \hline
    $\quad \Delta S = 0a\quad$ & $\quad \Delta L = 0, \pm 1 \quad$ & $\quad \Delta J = 0,\pm 1 \quad \left( 0 \rightarrow 0 forbidden \right) \quad$ \\
    \hline
  \end{tabular}
\end{table*}

\section{jj Coupling}

\begin{table*}[h]
  \centering
  \begin{tabular}{|Sc|Sc|Sc|}
    \hline
    Self-spin Angular Momentum & $s_1 = \dfrac{1}{2}  $ & $s_2 = \dfrac{1}{2} $  \\
    \hline
    Orbital Angular Momentum & $l_1$ & $l_2$  \\
    \hline
    Total Angular Momentum & $j_1 = l_1 \pm s_1$ & $j_2 = l_2 \pm s_2$  \\
    \hline
    Compound Total Angular Momentum & $J = | j_1 + j_2 |, | j_1 + j_2 | - 1, \dots , | j_1 - j_2 | $ & $P_J = \sqrt{J \left( J + 1 \right)} \hbar$  \\
    \hline
  \end{tabular}
\end{table*}




%%% Local Variables:
%%% mode: latex
%%% TeX-master: "Atomic_Physics"
%%% End:
