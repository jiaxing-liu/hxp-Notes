\documentclass{article}

% Chinese Support using xeCJK
% \usepackage{xeCJK}
% \setCJKmainfont{SimSun}

% Chinese Support using CTeX
\usepackage{ctex}

% Math Support
\usepackage{amsmath}
\usepackage{amsfonts}
\usepackage{amssymb}
\usepackage{wasysym}
\newcommand{\angstrom}{\text{\normalfont\AA}}

\usepackage{fancyhdr}

% Graphics Support
\usepackage{graphicx}
\usepackage{float}

% Reduced page margin
\usepackage{geometry}
\geometry{a4paper,scale=0.8}

\usepackage{caption}
\usepackage{subcaption}

% d and e should be math operators
\newcommand*{\dif}{\mathop{}\!\mathrm{d}}
\newcommand*{\md}{\mathop{}\!\mathrm{d}}
\newcommand*{\me}{\mathrm{e}}
\newcommand*{\mh}{\mathrm{h}}

% No indent for each paragraph
\usepackage{parskip}
\setlength{\parindent}{0cm}

% Bold style for Greek letters
\usepackage{bm}
\let\Oldmathbf\mathbf
\renewcommand{\mathbf}[1]{\boldsymbol{\Oldmathbf{#1}}}

% More space for dfrac in cell
\usepackage{cellspace}
\setlength{\cellspacetoplimit}{5pt}
\setlength{\cellspacebottomlimit}{5pt}

% SI units
\newcommand{\si}[1]{\  \mathrm{#1}}

% Multi-line author information
\usepackage{authblk}
\author{物理(4+4)1801 \quad 胡喜平 \quad 学号 U201811966}

\affil{网站 https://hxp.plus/ \quad 邮件 hxp201406@gmail.com}

\title{《量子力学教程》课后习题——第二章\ 波函数和薛定谔方程}

\pagestyle{fancy}
\fancyhf{}
\lhead{源码地址:https://github.com/hxp-plus/Notes/blob/master/Qumntum-Mechenics/Homework/}
\rfoot{第 \thepage 页}
\renewcommand{\headrulewidth}{1pt}
\renewcommand{\footrulewidth}{1pt}

\begin{document}

\maketitle\thispagestyle{fancy}


\paragraph{2.1}

证明在定态中,概率流密度与时间无关。

\paragraph{解}

\paragraph{2.2}

由下列两定态波函数计算概率流密度。

\begin{equation*}
  \begin{aligned}
    \psi_1 = \dfrac{1}{r} \exp \left[ i k r \right]
    \quad\quad \quad\quad \quad\quad 
    \psi_2 = \dfrac{1}{r} \exp \left[ - i k r \right] 
  \end{aligned}
\end{equation*}

从所得结果说明$\psi_1$表示向外传播的球面波,$\psi_2$表示向内(即向原点)传播的球面波。

\paragraph{解}

\paragraph{2.3}

一粒子在一维势场

\begin{equation*}
  \begin{aligned}
    U \left( x \right) =
  \end{aligned}
  \left\{
  \begin{aligned}
    & \infty && x<0 \\
    & 0 && 0 \leq x \leq a \\
    & \infty && x>a
  \end{aligned}
  \right.
\end{equation*}

中运动,求粒子的能级和对应的波函数。

\paragraph{解}

\paragraph{2.4}

证明下式中的归一化因子是$A'=\dfrac{1}{\sqrt{a}} $

\begin{equation*}
  \begin{aligned}
    \psi_n =
  \end{aligned}
  \left\{
  \begin{aligned}
    & A' \sin \dfrac{n\pi}{2a} \left( x + a \right) && \left| x \right| < a \\
    & 0 && \left| x \right| \geq a
  \end{aligned}
  \right.
\end{equation*}

\paragraph{解}

\paragraph{2.5}

求一维谐振子处在第一激发态时概率最大的位置。

\paragraph{解}

\paragraph{2.6}

在一维势场中运动的粒子,势能对原点对称:$U \left( x \right) = U \left( x \right)$,证明粒子的定态波函数具有确定的宇称。

\paragraph{解}

\paragraph{2.7}

一粒子在一维深势阱

\begin{equation*}
  \begin{aligned}
    U \left( x \right) =
  \end{aligned}
  \left\{
  \begin{aligned}
    & U_0 > 0 && \left| x \right| > a \\
    & 0 && \left| x \right| \leq a
  \end{aligned}
  \right.
\end{equation*}

中运动,求束缚态($0<E<U_0$)的能级所满足的方程。

\paragraph{解}

\paragraph{2.8}

分子间的范德瓦尔斯力所产生的势能可以近似地表示为

\begin{equation*}
  \begin{aligned}
    U \left( x  \right) = 
  \end{aligned}
  \left\{
  \begin{aligned}
    & \infty && x<0 \\
    & U_0 && 0 \leq x < a \\
    & - U_1 && a \leq x \leq b \\
    & 0 && b < x
  \end{aligned}
  \right.
\end{equation*}

求束缚态的能级所满足的方程。

\paragraph{解}

\end{document} 
