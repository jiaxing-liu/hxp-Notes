\documentclass{article}

% Chinese Support using xeCJK
% \usepackage{xeCJK}
% \setCJKmainfont{SimSun}

% Chinese Support using CTeX
\usepackage{ctex}

% Math Support
\usepackage{amsmath}
\usepackage{amsfonts}
\usepackage{amssymb}
\usepackage{wasysym}
\newcommand{\angstrom}{\text{\normalfont\AA}}

\usepackage{fancyhdr}

% Graphics Support
\usepackage{graphicx}
\usepackage{float}
\restylefloat{table}


% Reduced page margin
\usepackage{geometry}
\geometry{a4paper,scale=0.8}

\usepackage{caption}
\usepackage{subcaption}

% d and e should be math operators
\newcommand*{\dif}{\mathop{}\!\mathrm{d}}
\newcommand*{\md}{\mathop{}\!\mathrm{d}}
\newcommand*{\me}{\mathrm{e}}
\newcommand*{\mh}{\mathrm{h}}


\newcommand*{\Jmath}{J}

% No indent for each paragraph
% \usepackage{parskip}
% \setlength{\parindent}{0cm}

% Bold style for Greek letters
\usepackage{bm}
\let\Oldmathbf\mathbf
\renewcommand{\mathbf}[1]{\boldsymbol{\Oldmathbf{#1}}}

% More space for dfrac in cell
\usepackage{cellspace}
\setlength{\cellspacetoplimit}{5pt}
\setlength{\cellspacebottomlimit}{5pt}

% SI units
\newcommand{\si}[1]{\  \mathrm{#1}}

% Multi-line author information
\usepackage{authblk}
\author{物理(4+4)1801 \quad 胡喜平 \quad 学号 U201811966}

\affil{网站 https://hxp.plus/ \quad 邮件 hxp201406@gmail.com}

\title{《量子力学教程》课后习题——第三章\ 量子力学中的力学量}

\pagestyle{fancy}
\fancyhf{}
\lhead{源码地址:https://github.com/hxp-plus/Notes/blob/master/Qumntum-Mechenics/Homework/}
\rfoot{第 \thepage 页}
\renewcommand{\headrulewidth}{1pt}
\renewcommand{\footrulewidth}{1pt}

\begin{document}

\maketitle\thispagestyle{fancy}

\paragraph{3.1}

\begin{equation*}
  \begin{aligned}
    \overline{U} = \dfrac{1}{2} m \omega \overline{x^2} = \dfrac{1}{2} m \omega^2 \dfrac{\alpha}{\sqrt{\pi}} \int_{-\infty}^{+\infty} \me^{- \frac{1}{2}  \alpha^2 x^2} x^2 \me^{- \frac{1}{2}  \alpha^2 x^2} \md x =  \dfrac{1}{2} m \omega^2 \dfrac{\alpha}{\sqrt{\pi}} \cdot \dfrac{\sqrt{\pi}}{\alpha} \dfrac{1}{2\alpha^2} = \dfrac{1}{4} \dfrac{m \omega^2}{\alpha^2}       
  \end{aligned}
\end{equation*}

其中用到高斯积分公式

\begin{equation*}
  \begin{aligned}
    \int_{-\infty}^{+\infty} x^{2n} \me^{-\alpha x^2} \md x = \sqrt{\dfrac{\pi}{\alpha} } \dfrac{\left( 2n-1 \right)!!}{\left( 2\alpha \right)^n} 
  \end{aligned}
\end{equation*}

同理

\begin{equation*}
  \begin{aligned}
    \overline{T} &= \dfrac{\alpha}{\sqrt{\pi}} \int_{-\infty}^{+\infty} \me^{-\frac{1}{2} \alpha^2 x^2 } \dfrac{- \hbar}{2m} \dfrac{\md^2}{\md x^2} \me^{- \frac{1}{2} \alpha^2 x^2 } \md x
    = - \dfrac{\alpha \hbar}{2m\sqrt{\pi}} \int_{-\infty}^{+\infty} \me^{- \frac{1}{2} \alpha x^2} \cdot \alpha^2 \left( \alpha^2 x^2 - 1 \right) \cdot \me^{- \frac{1}{2} \alpha^2 x^2} \md x \\
    &= - \dfrac{\alpha^3 \hbar}{2 m \sqrt{\pi}} \int_{-\infty}^{+\infty} \left( \alpha^2 x^2 -1 \right) \me^{- \alpha^2 x^2} \md x
    = - \dfrac{\alpha^2 \hbar}{2m \sqrt{\pi}} \int_{- \infty}^{+ \infty} \left( \alpha^2 x^2 - 1 \right) \me^{- \alpha^2 x^2} \md \left( \alpha x \right) \\
    &= - \dfrac{\alpha^2 \hbar}{2m \sqrt{\pi}} \left[ \dfrac{\sqrt{\pi}}{2} - \sqrt{\pi} \right] 
    = \dfrac{1}{4} \dfrac{\hbar \alpha^2}{m}
    = \dfrac{\hbar \omega}{4} 
  \end{aligned}
\end{equation*}

动量的概率密度为

\begin{equation*}
  \begin{aligned}
    c \left( p \right) = \int_{-\infty}^{+\infty} \dfrac{1}{\sqrt{2\pi \hbar}} \me^{ipx/h} \sqrt{\dfrac{\alpha}{\pi^{1/2}} } \me^{- \frac{1}{2} \alpha^2 x^2} \md x = \sqrt{\dfrac{1}{\alpha \hbar \sqrt{\pi}} } \me^{- \frac{p^2}{2 \alpha^2 \hbar^2} } 
  \end{aligned}
\end{equation*}

动量概率的分布函数为

\begin{equation*}
  \begin{aligned}
    w \left( p \right) = \left| c \left( p \right) \right|^2 = \dfrac{1}{\alpha \hbar \sqrt{\pi}} \me^{- \frac{p^2}{\alpha^2 \hbar^2} } 
  \end{aligned}
\end{equation*}

\paragraph{3.2}

$r$的期望值为

\begin{equation*}
  \begin{aligned}
    \overline{r} = \iiint \psi \left( r \right) r \psi \left( r \right) \sin \theta r ^2 \md \theta \md r \md \phi = \dfrac{3}{2} a_0 
  \end{aligned}
\end{equation*}

势能$U$的期望值为

\begin{equation*}
  \begin{aligned}
    \overline{U} = \iint \md \Omega \int_0^{\infty} \psi \left( r \right) - \dfrac{e_s^2}{r} \psi \left( r \right) r^2 \md r = - \dfrac{e_s^2}{a_0} 
  \end{aligned}
\end{equation*}

动能$T$的期望值为

\begin{equation*}
  \begin{aligned}
    \overline{T} = \iint \md \Omega \int_0^{\infty} \left[ \psi \left( r \right) \dfrac{- \hbar}{2mr} \dfrac{\partial^2 }{\partial r^2} r \psi \left( r \right)   \right] r^2 \md r = \dfrac{e_s^2}{2 a_0} 
  \end{aligned}
\end{equation*}

在最概然半径处,径向概率取级值

\begin{equation*}
  \begin{aligned}
    \dfrac{\md}{\md r} \left[ w \left( r \right) \right] =
    \dfrac{\md }{\md r} \left[ R^2 \left( r \right) r^2 \right] = \dfrac{\md}{\md r} \left[ \dfrac{4\me^{-2r/a_0}}{a_0^3} r^2 \right] = 0
    \Rightarrow
    r = a_0
  \end{aligned}
\end{equation*}

动量分布的概率幅为

\begin{equation*}
  \begin{aligned}
    c \left( p \right) = \iiint \dfrac{1}{\left( \sqrt{2\pi \hbar} \right)^3} \me^{- i \frac{p \cdot r}{\hbar} } \psi \left( r \right) \md r 
  \end{aligned}
\end{equation*}

动量的概率密度为

\begin{equation*}
  \begin{aligned}
    w \left( p \right) = \left| c \left( p \right) \right|^2 = \dfrac{8 a_0^3 \hbar^5}{\pi^2 \left( \hbar^2 + a_0^2 p^2 \right)^4} 
  \end{aligned}
\end{equation*}

\paragraph{3.3}

概率流密度公式为

\begin{equation*}
  \begin{aligned}
    \vec{\Jmath} = \dfrac{i \hbar}{2m} \left( \psi \nabla \psi^{*} - \psi^{*} \nabla \psi \right) 
  \end{aligned}
\end{equation*}

其中

\begin{equation*}
  \begin{aligned}
    \nabla \psi_{nlm} = \left( -1 \right)^m N_{lm} \left[ e_r P_l^m \me^{im\phi} \dfrac{\partial R_{nl}}{\partial r} + e_{\theta} \dfrac{R_{nl} \me^{im\phi}}{r} \dfrac{\partial P_l^m}{r \sin \theta} + e_{\phi} \dfrac{R_{nl} P_l^m}{r \sin \theta}  \dfrac{\partial \me^{im\phi}}{\partial \phi}     \right]
  \end{aligned}
\end{equation*}

因此

\begin{equation*}
  \begin{aligned}
    \vec{\Jmath} \left( \vec{r}, t \right) =
    \dfrac{i \hbar}{2m_e } \left[ \left( N_{lm} R_{nl} P_l^m \right)^2 \left( \dfrac{-2 im}{r \sin \theta}  \right) e_\varphi \right]
    = \dfrac{\hbar m}{m_e r \sin \theta} \left| \psi_{nlm} \right|^2 e_{\varphi}
  \end{aligned}
\end{equation*}

即

\begin{equation*}
  \begin{aligned}
    &J_{er} = 0 \\
    &J_{e\varphi} =  \dfrac{\hbar m}{m_e r \sin \theta} \left| \psi_{nlm} \right|^2 e_{\varphi}
  \end{aligned}
\end{equation*}

\paragraph{3.4}

\begin{equation*}
  \begin{aligned}
    \md M = J_e r \md r \md \theta \cdot \pi r^2 \sin^2 \theta = - \dfrac{\pi e \hbar m}{m_e} w_{nl} r^2 \sin \theta \md r \md \varphi \md \theta 
  \end{aligned}
\end{equation*}

\begin{equation*}
  \begin{aligned}
    M = \iint \md M = - \dfrac{e \hbar m}{2m_e} 
  \end{aligned}
\end{equation*}

\paragraph{3.5}

转子的哈密顿算符为

\begin{equation*}
  \begin{aligned}
    H = L^2 / \left( 2I \right)
  \end{aligned}
\end{equation*}

定轴转动,薛定谔方程为

\begin{equation*}
  \begin{aligned}
    \dfrac{\hat{L}^2}{2I} \psi = - \dfrac{\hbar^2}{2 I} \dfrac{\md^2 \psi}{\md \varphi^2} = E \left( \varphi \right)  
  \end{aligned}
\end{equation*}

和一维粒子相同可以解出

\begin{equation*}
  \begin{aligned}
    E_m = \dfrac{m^2 \hbar^2}{2I} 
  \end{aligned}
\end{equation*}

定点转动时

\begin{equation*}
  \begin{aligned}
    - \dfrac{\hbar^2}{2I} \nabla^2 \psi = E \psi 
  \end{aligned}
\end{equation*}

边界条件为

\begin{equation*}
  \begin{aligned}
    \psi \left( \theta, \varphi + 2\pi \right) = \psi \left( \theta, \varphi \right)
  \end{aligned}
\end{equation*}

且在无穷远处有界,因此

\begin{equation*}
  \begin{aligned}
    E_l = \dfrac{l \left( l+1 \right) \hbar^2}{2I} 
  \end{aligned}
\end{equation*}

\paragraph{3.6}

函数是周期函数,只需要在一个周期内求动能和动量的期望

\begin{equation*}
  \begin{aligned}
    A^2 = \dfrac{1}{\int_0^{\pi/k} \left| \psi \left( x \right) \right|^2 \md x} = \dfrac{2k}{\pi} 
  \end{aligned}
\end{equation*}

\begin{equation*}
  \begin{aligned}
    \overline{p} = \int_0^{\pi/k} \psi^{*} \left( x \right) \left( - i \hbar \right) \dfrac{\md}{\md x} \psi \left( x \right) \md x = 0    
  \end{aligned}
\end{equation*}

\begin{equation*}
  \begin{aligned}
    \overline{T} = \overline{\dfrac{p^2}{2m} } =
    \int_0^{\pi/k} \psi^{*} \left( x \right) \dfrac{- \hbar^2}{2m} \dfrac{\md^2}{\md x^2} \psi \left( x \right) \md x = \dfrac{5\pi \hbar^2 k}{16m} A^2 = \dfrac{5 \hbar^2 k^2}{8m}    
  \end{aligned}
\end{equation*}




\end{document} 
