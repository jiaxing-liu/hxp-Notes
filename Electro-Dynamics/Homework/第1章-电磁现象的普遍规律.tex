\documentclass{article}

% Chinese Support using xeCJK
% \usepackage{xeCJK}
% \setCJKmainfont{SimSun}

% Chinese Support using CTeX
\usepackage{ctex}

% Math Support
\usepackage{amsmath}
\usepackage{amsfonts}
\usepackage{amssymb}
\usepackage{wasysym}
\newcommand{\angstrom}{\text{\normalfont\AA}}

\usepackage{fancyhdr}

% Graphics Support
\usepackage{graphicx}
\usepackage{float}

% Reduced page margin
\usepackage{geometry}
\geometry{a4paper,scale=0.8}

\usepackage{caption}
\usepackage{subcaption}

% d and e should be math operators
\newcommand*{\dif}{\mathop{}\!\mathrm{d}}
\newcommand*{\md}{\mathop{}\!\mathrm{d}}
\newcommand*{\me}{\mathrm{e}}
\newcommand*{\mh}{\mathrm{h}}
\newcommand*{\Jmath}{J}

% No indent for each paragraph
\usepackage{parskip}
\setlength{\parindent}{0cm}

% Bold style for Greek letters
\usepackage{bm}
\let\Oldmathbf\mathbf
\renewcommand{\mathbf}[1]{\boldsymbol{\Oldmathbf{#1}}}

% More space for dfrac in cell
\usepackage{cellspace}
\setlength{\cellspacetoplimit}{5pt}
\setlength{\cellspacebottomlimit}{5pt}

% SI units
\newcommand{\si}[1]{\  \mathrm{#1}}

% Multi-line author information
\usepackage{authblk}
\author{物理(4+4)1801 \quad 胡喜平 \quad 学号 U201811966}

\affil{网站 https://hxp.plus/ \quad 邮件 hxp201406@gmail.com}

\title{《电动力学》课后习题——第一章\ 电磁现象的基本规律}

\pagestyle{fancy}
\fancyhf{}
\lhead{源码地址:https://github.com/hxp-plus/Notes/blob/master/Eletro-Dynamics/Homework/}
\rfoot{第 \thepage 页}
\renewcommand{\headrulewidth}{1pt}
\renewcommand{\footrulewidth}{1pt}

\begin{document}

\maketitle\thispagestyle{fancy}

\paragraph{1.1}

根据算符$\nabla$的微分性与矢量性,推导下列公式:

\begin{equation}
  \begin{aligned}
    \label{eq:q1}
    \nabla \left( \vec{A} \cdot \vec{B} \right) = \vec{B} \times \left( \nabla \cdot \vec{A} \right) + \left( \vec{B} \cdot \nabla \right) \vec{A} + \vec{A} \times \left( \nabla \times \vec{B}  \right) 
  \end{aligned}
\end{equation}

\begin{equation}
  \label{eq:q2}
  \begin{aligned}
    \vec{A} \times \left( \nabla \times \vec{A} \right) = \dfrac{1}{2} \nabla A^2 - \left( \vec{A} \cdot \nabla \right) \vec{A} 
  \end{aligned}
\end{equation}

\paragraph{解}

\begin{equation}
  \label{eq:1}
  \begin{aligned}
    \nabla \left( \vec{A} \cdot \vec{B} \right) = \left( \partial_i \vec{e}_i \right) \left( A_j B_j  \right) = \left( A_j \partial_i B_j + B_j \partial_i A_j \right) \vec{e}_i
  \end{aligned}
\end{equation}

\begin{equation}
  \label{eq:2}
  \begin{aligned}
    \left( \vec{B} \cdot \nabla \right) \vec{A} = \left( B_i \vec{e}_i \cdot \partial_j \vec{e}_j  \right) \vec{A} = \left( \delta_{ij} B_i \partial_j \right) \vec{A} = \left( B_i \partial_i  \right) \left( A_j \vec{e}_j \right) = B_i \partial_i A_j \vec{e}_j
  \end{aligned}
\end{equation}

同理

\begin{equation}
  \label{eq:3}
  \begin{aligned}
    \left( \vec{A} \cdot \nabla \right) \vec{B} = A_i \partial_i B_j \vec{e}_j
  \end{aligned}
\end{equation}

\begin{equation}
  \label{eq:4}
  \begin{aligned}
    \vec{B} \times \left( \nabla \times \vec{A} \right) &= \vec{B} \times \left( \epsilon_{ijk} \partial_i A_j \vec{e}_k \right) = \epsilon_{mnl} B_m \left( \epsilon_{ijk} \partial_i A_j \vec{e}_k \right)_n \vec{e}_l = \epsilon_{mnl} B_m \epsilon_{ijn} \partial_i A_j \vec{e}_l = \epsilon_{lmn} \epsilon_{ijn} B_m \partial_i A_j \vec{e}_l\\ &= \left( B_m \partial_l A_m - B_m \partial_m A_l \right) \vec{e}_l
  \end{aligned}
\end{equation}

同理

\begin{equation}
  \label{eq:5}
  \begin{aligned}
    \vec{A} \times \left( \nabla \times \vec{B} \right) = \left( A_m \partial_l B_m - A_m \partial_m B_l \right) \vec{e}_l
  \end{aligned}
\end{equation}

式(\ref{eq:2})(\ref{eq:3})(\ref{eq:4})(\ref{eq:5})相加,显然等于式(\ref{eq:1}),因此式(\ref{eq:q1})得证。

\begin{equation}
  \label{eq:6}
  \begin{aligned}
    \vec{A} \times \left( \nabla \times \vec{A} \right) &= \vec{A} \times \left[ \left( \partial_i \vec{e}_i \right) \times \left( A_j \vec{e}_j \right) \right] = \vec{A} \times \left( \epsilon_{ijk} \partial_i A_j \vec{e}_k \right) = \left( A_l \vec{e}_l \right) \times \left( \epsilon_{ijk} \partial_{ijk} \partial_i A_j \vec{e}_k \right) \\ &= \epsilon_{ijk} \epsilon_{lkn} A_l \partial_i A_j \vec{e}_n = \epsilon_{ijk} \epsilon_{nlk} A_l \partial_i A_j \vec{e}_n = \left( \delta_{in} \delta_{jl} - \delta_{il} \delta_{jn} \right) A_l \partial_i A_j \vec{e}_n \\ &= A_j \partial_i A_j \vec{e}_i 
  \end{aligned}
\end{equation}

\begin{equation}
  \label{eq:7}
  \begin{aligned}
    \left( \vec{A} \cdot \nabla \right) \vec{A} = \left( A_i \partial_i \right) \left( A_j \vec{e}_j \right) = A_j \partial_i A_j \vec{e}_j
  \end{aligned}
\end{equation}

显然,式(\ref{eq:6})和(\ref{eq:7})是相等的,得证。

\paragraph{1.2}

设$u$是空间坐标$x,y,z$的函数,证明:

\begin{equation*}
  \begin{aligned}
    \nabla f \left( u \right) = \dfrac{\md f}{\md u} \nabla u 
  \end{aligned}
\end{equation*}

\begin{equation*}
  \begin{aligned}
    \nabla \cdot \vec{A} \left( u \right) = \nabla u \cdot \dfrac{\md \vec{A}}{\md u} 
  \end{aligned}
\end{equation*}

\begin{equation*}
  \begin{aligned}
    \nabla \times \vec{A} \left( u \right) = \nabla u \times \dfrac{\md \vec{A}}{\md u} 
  \end{aligned}
\end{equation*}

\paragraph{解}

\begin{equation*}
  \begin{aligned}
    \nabla f = \partial_i f_i \vec{e}_i = \dfrac{\partial}{\partial x_i} f_i \vec{e}_i = \dfrac{\partial f_i}{\partial u} \dfrac{\partial u}{\partial x_i} \vec{e}_i = \dfrac{\md f}{\md u} \nabla u     
  \end{aligned}
\end{equation*}

\begin{equation*}
  \begin{aligned}
    \nabla \cdot \vec{A} = \partial_i A_i = \dfrac{\partial A_i}{\partial x_i} = \dfrac{\partial A_i}{\partial _u} \dfrac{\partial u}{\partial x_i} = \nabla u \cdot \dfrac{\md \vec{A}}{\md u}   
  \end{aligned}
\end{equation*}

\begin{equation*}
  \begin{aligned}
    \nabla \times \vec{A} = \epsilon_{ijk} \partial_i A_j \vec{e}_k = \epsilon_{ijk} \dfrac{\partial u}{\partial x_i} \dfrac{\partial A_j}{\partial u} \vec{e}_k = \nabla u \times \dfrac{\md \vec{A}}{\md u}  
  \end{aligned}
\end{equation*}

\paragraph{1.3}

设$r=\sqrt{\left( x - x' \right)^2 + \left( y - y' \right)^2 + \left( z - z' \right)^2}$为源点$x'$到$x$的距离,$\vec{r}$的方向规定为源点指向场点。

(1)证明以下结果,并体会对源变数求微商($\nabla' = \vec{e}_i \dfrac{\partial}{\partial x_i'} $)和对场变数求微商($\nabla = \vec{e}_i \dfrac{\partial }{\partial x_i} $)的关系。

\begin{equation*}
  \begin{aligned}
    \nabla \vec{r} = - \nabla' \vec{r} = \dfrac{\vec{r}}{r} 
  \end{aligned}
\end{equation*}

\begin{equation*}
  \begin{aligned}
    \nabla \dfrac{1}{r} = - \nabla' \dfrac{1}{r} = - \dfrac{\vec{r}}{r^3}   
  \end{aligned}
\end{equation*}

\begin{equation*}
  \begin{aligned}
    \nabla \times \dfrac{\vec{r}}{r^3} = 0 
  \end{aligned}
\end{equation*}

\begin{equation*}
  \begin{aligned}
    \nabla \cdot \dfrac{\vec{r}}{r^3} = - \nabla' \cdot \dfrac{\vec{r}}{r^3} = 0  
  \end{aligned}
\end{equation*}

\paragraph{解}

因为

\begin{equation*}
  \begin{aligned}
    \nabla = \partial_i = \dfrac{\partial}{\partial x_i} = \dfrac{\partial}{\partial x'_i} \dfrac{\partial x'_i}{\partial x_i} = - \dfrac{\partial}{\partial x'_i} = - \nabla   
  \end{aligned}
\end{equation*}

所以

\begin{equation*}
  \begin{aligned}
    \nabla r = - \nabla' r
  \end{aligned}
\end{equation*}

\begin{equation*}
  \begin{aligned}
    \nabla \dfrac{1}{r} = - \nabla' \dfrac{1}{r}
  \end{aligned}
\end{equation*}

\begin{equation*}
  \begin{aligned}
    \nabla \times \dfrac{\vec{r}}{r^3} = 0 
  \end{aligned}
\end{equation*}

\begin{equation*}
  \begin{aligned}
    \nabla \cdot \dfrac{\vec{r}}{r^3} = - \nabla' \cdot \dfrac{\vec{r}}{r^3} 
  \end{aligned}
\end{equation*}

只需要证明

\begin{equation}
  \label{eq:a311}
  \begin{aligned}
    \nabla \vec{r} = \dfrac{\vec{r}}{r} 
  \end{aligned}
\end{equation}

\begin{equation}
  \label{eq:a312}
  \begin{aligned}
    \nabla \dfrac{1}{r} = - \dfrac{\vec{r}}{r^3}   
  \end{aligned}
\end{equation}

\begin{equation}
  \label{eq:a313}
  \begin{aligned}
    \nabla \times \dfrac{\vec{r}}{r^3} = 0 
  \end{aligned}
\end{equation}

\begin{equation}
  \label{eq:a314}
  \begin{aligned}
    \nabla \cdot \dfrac{\vec{r}}{r^3}  = 0  
  \end{aligned}
\end{equation}

对于(\ref{eq:a311})

\begin{equation*}
  \begin{aligned}
    \nabla r = \partial_i \sqrt{ \Sigma \left( x_i - x_i' \right)^2} \vec{e}_i = \dfrac{x_i}{r} \vec{e}_i = \dfrac{\vec{r}}{r}  
  \end{aligned}
\end{equation*}

对于(\ref{eq:a312})

\begin{equation*}
  \begin{aligned}
    \nabla \dfrac{1}{r} = - \dfrac{1}{r^2} \nabla r = - \dfrac{\vec{r}}{r^3} 
  \end{aligned}
\end{equation*}

对于(\ref{eq:a313})

\begin{equation*}
  \begin{aligned}
    \nabla \times \dfrac{\vec{r}}{r^3} = \epsilon_{ijk} \partial_i \left( \dfrac{\vec{r}}{r^3}  \right)_j \vec{e}_k = \epsilon_{ijk} \partial_i \left( \dfrac{x_j}{r^3}  \right) \vec{e}_k = 0
  \end{aligned}
\end{equation*}

对于(\ref{eq:a314})

\begin{equation*}
  \begin{aligned}
    \nabla \cdot \dfrac{\vec{r}}{r^3}  &= \partial_i \left( \dfrac{x_i}{r^3}  \right) = \dfrac{r^3 \partial_i x_i - x_i \partial_i r^3}{r^6} = \dfrac{r^3 \partial_i x_i - 3 r^2 x_i \partial_i r}{r^6 } = \dfrac{r^3 - 3 r^2 x_i \dfrac{x_i}{r} }{r^6} = \dfrac{r^3 - 3 r x_i^2  }{r^6}  \\
    &= \dfrac{r^3 - 3 r x_1^2  }{r^6} + \dfrac{r^3 - 3 r x_2^2  }{r^6} + \dfrac{r^3 - 3 r x_3^2  }{r^6} = \dfrac{3 r^3 - 3r \left( x_1^2 + x_2^2 + x_3^2 \right)}{r^6} = 0
  \end{aligned}
\end{equation*}

由于分母上有$r$,当$r=0$时,不一定成立。

(2)求$\nabla \cdot \vec{r}$,$\nabla \times \vec{r}$,$\left( \vec{a} \cdot \nabla \right) \vec{r}$,$\nabla \left( \vec{a} \cdot \vec{r} \right)$,$\nabla \cdot \left[ \vec{E}_0 \sin \left( \vec{k} \cdot \vec{r} \right) \right]$,$\nabla \times \left[ \vec{E}_0 \sin \left( \vec{k} \cdot \vec{r} \right) \right]$,$\vec{a}$、$\vec{k}$、$\vec{E}_0$是常矢量。

\paragraph{解}

\begin{equation*}
  \begin{aligned}
    \nabla \cdot \vec{r} = \partial_i r_i = 3
  \end{aligned}
\end{equation*}

\begin{equation*}
  \begin{aligned}
    \nabla \times \vec{r} = \epsilon_{ijk} \partial_i x_j \vec{e}_k = 0
  \end{aligned}
\end{equation*}

\begin{equation*}
  \begin{aligned}
    \left( \vec{a} \cdot \nabla \right) \vec{r} = a_i \partial_i x_i \vec{e}_i = a_i \vec{e}_i = \vec{a}
  \end{aligned}
\end{equation*}

\begin{equation*}
  \begin{aligned}
    \nabla \left( \vec{a} \cdot \vec{r} \right) = \partial_i \left( a_j r_j \right) \vec{e}_i = \left[ a_j \partial_i x_j + x_j \partial_i a_j \right] = a_j \partial_i x_j = a_i \partial_i x_i \vec{e}_i = a_i \vec{e}_i = \vec{a}
  \end{aligned}
\end{equation*}

\begin{equation*}
  \begin{aligned}
    \nabla \cdot \left[ \vec{E}_0 \sin \left( \vec{k} \cdot \vec{r} \right) \right] = \partial_i E_{0i} \sin \left( k_i x_i \right) = E_{0i} \cos \left( k_i x_i \right) k_i = \vec{k} \cdot \vec{E}_0 \cos \left( \vec{k} \cdot \vec{r} \right) 
  \end{aligned}
\end{equation*}

\begin{equation*}
  \begin{aligned}
    \nabla \times \left[ \vec{E}_0 \sin \left( \vec{k} \cdot \vec{r} \right) \right] = \epsilon_{ijk} \partial_i E_{0j} \sin \left( k_j x_j \right) \vec{e}_k = \epsilon_{ijk} E_{0j} \cos \left( k_m x_m \right) k_i \vec{e}_k = \vec{k} \times \vec{E}_0 \cos \left( \vec{k} \cdot \vec{r} \right)
  \end{aligned}
\end{equation*}

\paragraph{1.4}

利用高斯定理证明

\begin{equation*}
  \begin{aligned}
    \int_V \md V \nabla \times \vec{f} = \oint_S \md \vec{S} \times \vec{f} 
  \end{aligned}
\end{equation*}

利用斯托克斯定理证明

\begin{equation*}
  \begin{aligned}
    \int_S \md \vec{S} \times \nabla \varphi = \oint_L \varphi \md \vec{l}
  \end{aligned}
\end{equation*}

\paragraph{解}

引入常矢量$\vec{c}$

\begin{equation*}
  \begin{aligned}
    \int_V \md V \nabla \times \vec{f} \cdot \vec{c} = \int_V \vec{c} \cdot \left( \nabla \times \vec{f} \right) \md V = \int_V \nabla \cdot \left( \vec{f} \times \vec{c} \right) \md V = \oint_S \left( \vec{f} \times \vec{c} \right) \cdot \md \vec{S} = \oint_S \md \vec{S} \times \vec{f} \cdot \vec{c}
  \end{aligned}
\end{equation*}

\begin{equation*}
  \begin{aligned}
    \int_S \md \vec{S} \times \nabla \varphi \cdot \vec{c} = \int_S \nabla \varphi \times \vec{c} \cdot \md \vec{S} = \int_L \nabla \times \left( \varphi \vec{c} \right) \cdot \md \vec{S} = \oint_L \varphi \vec{c} \cdot \md \vec{l} = \oint_L \varphi \md \vec{l} \cdot \vec{c}
  \end{aligned}
\end{equation*}

因为$\vec{c}$是任意的

\begin{equation*}
  \begin{aligned}
    \int_V \md V \nabla \times \vec{f} = \oint_S \md \vec{S} \times \vec{f} 
  \end{aligned}
\end{equation*}

\begin{equation*}
  \begin{aligned}
    \int_S \md \vec{S} \times \nabla \varphi = \oint_L \varphi \md \vec{l}
  \end{aligned}
\end{equation*}

\paragraph{1.5}

已知一个电荷系统的电偶极矩为

\begin{equation*}
  \begin{aligned}
    \vec{p} \left( t \right) = \int_V \rho \left( \vec{x}', t \right) \vec{x}' \md V'
  \end{aligned}
\end{equation*}

利用电荷守恒定律$\nabla \cdot \vec{\Jmath} + \dfrac{\partial \rho}{\partial t} = 0 $证明

\begin{equation*}
  \begin{aligned}
    \dfrac{\md \vec{p}}{\md t} = \int_V \vec{\Jmath} \left( \vec{x}', t \right) \md V' 
  \end{aligned}
\end{equation*}

\paragraph{解}

\begin{equation*}
  \begin{aligned}
    \dfrac{\md \vec{p}}{\md t} &= \int_V \dfrac{\md \rho \left( \vec{x}', t \right) }{\md t} \vec{x}' \md V' = - \int_V \left( \nabla' \cdot \vec{\Jmath} \right) \vec{x}' \md V' = - \int_V \left( \nabla' \cdot \vec{\Jmath} \vec{x}' \right) - \vec{\Jmath} \left( \nabla' \vec{x}' \right) \md V' \\
    &= - \oint_S \vec{\Jmath} \vec{x}' \cdot \md \vec{S} + \int_V \vec{\Jmath} \md V' = \int_V \vec{\Jmath} \md V'
  \end{aligned}
\end{equation*}

\paragraph{1.6}

若$\vec{m}$是常矢量,证明除$R=0$以外,矢量$\vec{A} = \dfrac{\vec{m} \times \vec{R}}{R^3} $的旋度等于标量$\varphi = \dfrac{\vec{m} \cdot \vec{R}}{R^3} $的梯度的负值,即

\begin{equation*}
  \begin{aligned}
    \nabla \times \vec{A} = - \nabla \varphi
  \end{aligned}
\end{equation*}

\paragraph{解}

\begin{equation*}
  \begin{aligned}
    \nabla \times \vec{A} &= \nabla \times \left( \vec{m} \times \dfrac{\vec{R}}{R^3}  \right) = \left( \dfrac{\vec{R}}{R^3} \cdot \nabla  \right) \vec{m} + \left( \nabla \cdot \dfrac{\vec{R}}{R^3}  \right) \vec{m} - \left( \vec{m} \cdot \nabla \right)\dfrac{\vec{R}}{R^3} - \left( \nabla \cdot \vec{m} \right) \dfrac{\vec{R}}{R^3} \\
    &= \left( \nabla \cdot \dfrac{\vec{R}}{R^3}  \right) \vec{m} - \left( \vec{m} \cdot \nabla \right) \dfrac{\vec{R}}{R^3} = - \left( \vec{m} \cdot \nabla \right) \dfrac{\vec{R}}{R^3}  
  \end{aligned}
\end{equation*}

\begin{equation*}
  \begin{aligned}
    - \nabla \varphi &= - \nabla \left( \dfrac{\vec{m} \cdot \vec{R}}{R^3}  \right) = - \vec{m} \times \left( \nabla \times \dfrac{\vec{R}}{R^3}  \right) - \left( \vec{m} \cdot \nabla \right) \dfrac{\vec{R}}{R^3} - \dfrac{\vec{R}}{R^3} \times \left( \nabla \times \vec{m} \right) - \left( \dfrac{\vec{R}}{R^3} \cdot \nabla  \right) \vec{m} \\
    &= - \vec{m} \times \left( \nabla \times \dfrac{\vec{R}}{R^3}  \right) - \left( \vec{m} \cdot \nabla \right) \dfrac{\vec{R}}{R^3} = - \left( \vec{m} \cdot \nabla \right) \dfrac{\vec{R}}{R^3}  
  \end{aligned}
\end{equation*}

得证

\paragraph{1.7}

有一内外半径为$r_1$和$r_2$的空心介质球,介质的电容率为$\varepsilon$,是介质内均匀带静自由电荷密度$\rho_f$,求:

(1)空间各点的电场

(2)极化体电荷和极化面电荷分布

\paragraph{解}

在$r < r_1$时

\begin{equation*}
  \begin{aligned}
    \oint_S \vec{D} \cdot \md \vec{S} = \int_V \rho_f \md V
    \Rightarrow
    4\pi r^2 D = 0
    \Rightarrow
    \vec{D} = 0
    \Rightarrow
    \vec{E} = 0
  \end{aligned}
\end{equation*}

在$r_1 < r < r_2$时

\begin{equation*}
  \begin{aligned}
    \oint_S \vec{D} \cdot \md \vec{S} = \int_V \rho_f \md V
    \Rightarrow
    4\pi r^2 D = \dfrac{4}{3} \pi \left( r^3 - r_1^3 \right) \rho_f
    \Rightarrow
    \vec{D} = \dfrac{r^3 - r_1^3}{3 r^3} \rho_f \vec{r} 
    \Rightarrow
    \vec{E} = \dfrac{r^3 - r_1^3}{3 \varepsilon r^3} \rho_f \vec{r}
  \end{aligned}
\end{equation*}

在$r > r_2$时

\begin{equation*}
  \begin{aligned}
    \oint_S \vec{D} \cdot \md \vec{S} = \int_V \rho_f \md V
    \Rightarrow
    4\pi r^2 D = \dfrac{4}{3} \pi \left( r_2^3 - r_1^3 \right) \rho_f
    \Rightarrow
    \vec{D} = \dfrac{r_2^3 - r_1^3}{3 r^3} \rho_f \vec{r}
    \Rightarrow
    \vec{D} = \dfrac{r_2^3 - r_1^3}{3 \varepsilon_0 r^3} \rho_f \vec{r}
  \end{aligned}
\end{equation*}

在$r_1 <r < r_2$时

\begin{equation*}
  \begin{aligned}
    D &= \varepsilon_0 E + P
    \Rightarrow
    P = D - \varepsilon_0 E = D \left( 1 - \dfrac{\varepsilon_0}{\varepsilon}  \right)
    \Rightarrow
    \vec{P} = \left( 1 - \dfrac{\varepsilon_0}{\varepsilon}  \right) \dfrac{r^3 - r_1^3}{3r^3} \rho_f \vec{r} \\
    &\Rightarrow
    \rho_P = - \nabla \cdot \vec{P} = - \left( 1 - \dfrac{\varepsilon_0}{\varepsilon}  \right) \rho_f
  \end{aligned}
\end{equation*}

在$r = r_2$时

\begin{equation*}
  \begin{aligned}
    \sigma_P = - \vec{e}_n \cdot \left( \vec{P}_3 - \vec{P}_2 \right) = \left( 1 - \dfrac{\varepsilon_0}{\varepsilon}  \right) \dfrac{r_2^3 - r_1^3}{3 r_2^2} \rho_f  
  \end{aligned}
\end{equation*}

在$r = r_1$时

\begin{equation*}
  \begin{aligned}
    \sigma_P = - \vec{e}_n \cdot \left( \vec{P}_2 - \vec{P}_1 \right) = 0
  \end{aligned}
\end{equation*}

\paragraph{1.8}

内外半径分别为$r_1$和$r_2$的无穷长中空导体圆柱,沿轴向流有恒定均匀自由电流$\vec{\Jmath}_f$,导体磁导率为$\mu$,求磁感应强度和磁化电流

\paragraph{解}

在$r<r_1$时

\begin{equation*}
  \begin{aligned}
    \oint \vec{H} \cdot \md \vec{l} = I_f + \dfrac{\md}{\md t} \int_S \vec{D} \cdot \md \vec{S}
    \Rightarrow
    \vec{H} = 0
    \Rightarrow
    \vec{B} = 0
  \end{aligned}
\end{equation*}

在$r_1<r<r_2$时

\begin{equation*}
  \begin{aligned}
    \oint \vec{H} \cdot \md \vec{l} = I_f + \dfrac{\md}{\md t} \int_S \vec{D} \cdot \md \vec{S}
    \Rightarrow
    2\pi r H = \pi \left( r^2 - r_1^2 \right) J_f
    \Rightarrow
    \vec{H} = \dfrac{r^2 - r_1^2}{2 r^2 } \vec{\Jmath}_f \times \vec{r} 
    \Rightarrow
    \vec{B} = \mu \dfrac{r^2 - r_1^2}{2 r^2 } \vec{\Jmath}_f \times \vec{r}
  \end{aligned}
\end{equation*}

在$r>r_2$时

\begin{equation*}
  \begin{aligned}
    \oint \vec{H} \cdot \md \vec{l} = I_f + \dfrac{\md}{\md t} \int_S \vec{D} \cdot \md \vec{S}
    \Rightarrow
    2\pi r H = \pi \left( r_2^2 - r_1^2 \right) J_f
    \Rightarrow
    \vec{H} = \dfrac{r_2^2 - r_1^2}{2 r^2 } \vec{\Jmath}_f \times \vec{r} 
    \Rightarrow
    \vec{B} = \mu_0 \dfrac{r_2^2 - r_1^2}{2 r^2 } \vec{\Jmath}_f \times \vec{r}
  \end{aligned}
\end{equation*}

当$r_1 < r < r_2$时

\begin{equation*}
  \begin{aligned}
    \vec{\Jmath}_m = \nabla \times \vec{M} = \left( \dfrac{\mu}{\mu_0} - 1  \right) \nabla \times \vec{H} = \left( \dfrac{\mu}{\mu_0} - 1  \right) \vec{\Jmath}_f
  \end{aligned}
\end{equation*}

当$r=r_2$时

\begin{equation*}
  \begin{aligned}
    \vec{\alpha}_M = \vec{e}_r \times \left( \vec{M}_3 - \vec{M}_2 \right) = - \left( \dfrac{\mu}{\mu_0} - 1  \right) \vec{e}_r \times \vec{H}_3 = - \left( \dfrac{\mu}{\mu_0} - 1  \right) \dfrac{r_2^2 - r_1^2}{ 2 r_2^2} \vec{\Jmath}_f 
  \end{aligned}
\end{equation*}

当$r=r_1$时

\begin{equation*}
  \begin{aligned}
    \vec{\alpha}_M = \vec{e}_r \times \left( \vec{M}_2 - \vec{M}_1 \right) = 0
  \end{aligned}
\end{equation*}


\paragraph{1.9}

证明均匀介质内部的极化电荷体密度$\rho_P$总是等于自由电荷体密度$\rho_f$的$- \left( 1 - \dfrac{\varepsilon_0}{\varepsilon}  \right)$倍

\paragraph{解}

\begin{equation*}
  \begin{aligned}
    \varepsilon E = D \Rightarrow \nabla \cdot \varepsilon E = \nabla \cdot D \Rightarrow \varepsilon \varepsilon_0 \nabla \cdot E = \varepsilon_0 \nabla \cdot D \Rightarrow \varepsilon \left( \rho_P + \rho_f \right) = \varepsilon_0 \rho_f \Rightarrow \rho_P = - \left( 1 - \dfrac{\varepsilon_0}{\varepsilon}  \right) \rho_f
  \end{aligned}
\end{equation*}

\paragraph{1.10}

证明两个闭合的恒定电流圈之间的相互作用力大小相等,方向相反(但两个电流元之间的相互作用力一般不服从牛顿第三定律)

\paragraph{解}

设两个电流圈的电流为$I_1$和$I_2$

\begin{equation*}
  \begin{aligned}
    \vec{F}_{12}
    &= \oint_{L2} I_2 \md \vec{l}_2 \times \dfrac{\mu_0}{4\pi} \oint_{L1} \dfrac{I_1 \md \vec{l}_1 \times \vec{r}_{12}}{r_{12}^3}
    = \dfrac{\mu_0 I_1 I_2}{4 \pi} \oint_{L2} \oint_{L1} \dfrac{\md \vec{l}_2 \times \md \vec{l}_1 \times \vec{r}_{12}}{r_{12}^3}\\
    &= \dfrac{\mu_0 I_1 I_2}{4 \pi} \oint_{L2} \oint_{L1} \dfrac{\left( \md \vec{l}_2 \cdot \vec{r}_{12} \right) \md \vec{l}_1 - \left( \md \vec{l}_2 \cdot \md \vec{l}_1 \right) \vec{r}_{12}}{r_{12}^3}
  \end{aligned}
\end{equation*}

因为

\begin{equation*}
  \begin{aligned}
    \oint_{L1} \oint_{L2} \dfrac{\left( \md \vec{l}_2 \cdot \vec{r}_{12} \right) \md \vec{l}_1}{r_{12}^3}
    = \oint_{L1} \left[ \oint_{L2} \dfrac{\vec{r}_{12}}{r_{12}^3} \cdot \md \vec{l}_2 \right] \md \vec{l}_1
    = \oint_{L1} \left[ \oint_{S2} \nabla \times \dfrac{\vec{r}_{12}}{r_{12}^3} \cdot \md \vec{S}_2 \right] \md \vec{l}_1
    = 0
  \end{aligned}
\end{equation*}

所以

\begin{equation*}
  \begin{aligned}
    \vec{F}_{12} = - \dfrac{\mu_0 I_1 I_2}{4 \pi} \oint_{L2} \oint_{L1} \dfrac{\md \vec{l}_2 \cdot \md \vec{l}_1}{r^3} \vec{r}_{12}
  \end{aligned}
\end{equation*}

同理

\begin{equation*}
  \begin{aligned}
    \vec{F}_{21} = - \dfrac{\mu_0 I_2 I_1}{4 \pi} \oint_{L1} \oint_{L2} \dfrac{\md \vec{l}_1 \cdot \md \vec{l}_2}{r^3} \vec{r}_{21}
  \end{aligned}
\end{equation*}

其中$r_{12} = - r_{21}$,因此

\begin{equation*}
  \begin{aligned}
    \vec{F}_{21} = - \vec{F}_{12}
  \end{aligned}
\end{equation*}

\paragraph{1.11}

平行班电容器内有两层介质,它们的厚度分别为$l_1$和$l_2$,电容率为$\varepsilon_1$和$\varepsilon_2$,在两板接上电动势为$\mathcal{E}$的电池,求

(1)电容器两板的自由电荷面密度$\omega_f$

(2)介质分界面上的自由电荷面密度$\omega_f$

若介质是漏电的,电导率分别为$\sigma_1$和$\sigma_2$,当电流达到恒定时,上述两问题结果如何?

\paragraph{解}

边值关系

\begin{equation*}
  \begin{aligned}
    \omega_f = \vec{e}_n \cdot \left( \vec{D}_2 - \vec{D}_1 \right)
  \end{aligned}
\end{equation*}

其中

\begin{equation}
  \label{eq:a1111}
  \begin{aligned}
    \mathcal{E} = l_1 \dfrac{D_1}{\varepsilon_1} + l_2 \dfrac{D_2}{\varepsilon_2}   
  \end{aligned}
\end{equation}

对于两个极板

\begin{equation}
  \label{eq:a1112}
  \begin{aligned}
    \omega_{f1} = D_1
    \quad\quad
    \omega_{f2} = - D_2
  \end{aligned}
\end{equation}

对于介质分界面

\begin{equation}
  \label{eq:a1113}
  \begin{aligned}
    \omega_{f3} = D_2 - D_1 = 0
  \end{aligned}
\end{equation}

由式(\ref{eq:a1111})(\ref{eq:a1112})(\ref{eq:a1113})可得

\begin{equation*}
  \begin{aligned}
    \omega_{f1} = \dfrac{\mathcal{E}}{\dfrac{l_1}{\varepsilon_1} + \dfrac{l_2}{\varepsilon_2}  } = - \omega_{f2} 
  \end{aligned}
\end{equation*}

介质漏电时,设电流密度$J$,式(\ref{eq:a1111})应当改为

\begin{equation*}
  \begin{aligned}
    \mathcal{E} = l_1 E_1 + l_2 E_2 = \left( \dfrac{l_1}{\sigma_1} + \dfrac{l_2}{\sigma_2}   \right) J
  \end{aligned}
\end{equation*}

同时

\begin{equation*}
  \begin{aligned}
    D_1 = \varepsilon_1 \dfrac{J}{\sigma_1}
    \quad\quad
    D_2 = \varepsilon_2 \dfrac{J}{\sigma_2}  
  \end{aligned}
\end{equation*}

解得

\begin{equation*}
  \begin{aligned}
    \omega_{f1} = \dfrac{\varepsilon_1 \sigma_2}{\sigma_2 l_1 + \sigma_1 l_2} \mathcal{E}
  \end{aligned}
\end{equation*}

\begin{equation*}
  \begin{aligned}
    \omega_{f2} = - \dfrac{\varepsilon_2 \sigma_1}{\sigma_2 l_1 + \sigma_1 l_2} \mathcal{E}
  \end{aligned}
\end{equation*}

\begin{equation*}
  \begin{aligned}
    \omega_{f3} = \dfrac{\varepsilon_2 \sigma_1 - \varepsilon_1 \sigma_2}{\sigma_2 l_1 + \sigma_1 l_2} 
  \end{aligned}
\end{equation*}

\paragraph{1.12}

证明

(1)当两种绝缘介质的分界面上不带自由电荷时,电场线的曲折满足

\begin{equation*}
  \begin{aligned}
    \dfrac{\tan \theta_2}{\tan \theta_1} = \dfrac{\varepsilon_2}{\varepsilon_1}  
  \end{aligned}
\end{equation*}

其中$\varepsilon_1$和$\varepsilon_2$分别为两种介质的介电常数,$\theta_1$和$\theta_2$分别为界面两侧电场线与法线的夹角。

(2)当两种导电介质内流有恒定电流时,分界面上电场线的曲折满足:

\begin{equation*}
  \begin{aligned}
    \dfrac{\tan \theta_1}{\tan \theta_2} = \dfrac{\sigma_2}{\sigma_1}  
  \end{aligned}
\end{equation*}

边值关系为

\begin{equation*}
  \begin{aligned}
    \varepsilon_1 E_{1n} = \varepsilon_2 E_{2n}
  \end{aligned}
\end{equation*}

\begin{equation*}
  \begin{aligned}
    E_{1t} = E_{2t}
  \end{aligned}
\end{equation*}

即

\begin{equation*}
  \begin{aligned}
    \varepsilon_1 E_1 \sin \theta_1 = \varepsilon_2 E_2 \sin \theta_2
  \end{aligned}
\end{equation*}

\begin{equation*}
  \begin{aligned}
    E_1 \cos \theta_1 = E_2 \cos \theta_2
  \end{aligned}
\end{equation*}

因此

\begin{equation*}
  \begin{aligned}
    \dfrac{\tan \theta_1}{\tan \theta_2} = \dfrac{\varepsilon_2}{\varepsilon_1}  
  \end{aligned}
\end{equation*}

当有恒定电流时

\begin{equation*}
  \begin{aligned}
    \sigma_1 E_{1n} = J_n
    \quad\quad
    \sigma_2 E_{2n} = J_n
    \quad\quad
    E_{1t} = E_{2t}
  \end{aligned}
\end{equation*}

所以

\begin{equation*}
  \begin{aligned}
     \dfrac{\tan \theta_1}{\tan \theta_2} = \dfrac{\sigma_2}{\sigma_1}  
  \end{aligned}
\end{equation*}

\paragraph{1.13}

试用边值关系证明:在绝缘介质与导体的分界面上,在静电情况下,导体外表面的电场线总是垂直于导体表面。在恒定电流情况下,导体内表面电场线总是平行于导体表面。

\paragraph{解}

由边值关系

\begin{equation*}
  \begin{aligned}
    D_{2n} = \sigma_f
    \quad\quad
    E_{2t} = 0
  \end{aligned}
\end{equation*}

导体外部

\begin{equation*}
  \begin{aligned}
    \vec{E}_2 = \dfrac{\vec{D}_2}{\varepsilon} = \dfrac{\sigma_f}{\varepsilon} \vec{e}_n  
  \end{aligned}
\end{equation*}

介质界面两侧法向电流应当是连续的

\begin{equation*}
  \begin{aligned}
    J_{1n} = J_{2n}
  \end{aligned}
\end{equation*}

所以介质内法向电场

\begin{equation*}
  \begin{aligned}
    E_{2n} = \dfrac{J_{2n}}{\sigma_2} = 0 
  \end{aligned}
\end{equation*}

\paragraph{1.14}

内外半径分别为$a$和$b$的无限长圆柱型电容器,单位长度荷电为$\lambda_f$,板间填充电导率为$\sigma$的非磁性物质

(1)证明在介质中任何一点的传导电流与位移电流完全抵消,因此内部无磁场。

(2)求$\lambda_f$随时间的衰减规律。

(3)求轴与相距为$r$的地方能量耗散功率密度。

(4)求长度为$l$的一段介质总的能量耗散功率,并证明它等于这段的静电减少率。

\paragraph{解}

在介质中有

\begin{equation*}
  \begin{aligned}
    \vec{D} = \dfrac{\lambda_f}{2 \pi r} \vec{e}_r
    \quad\quad
    \vec{E} = \dfrac{\vec{D}}{\varepsilon} = \dfrac{\lambda_f}{2 \pi \varepsilon r} \vec{e}_r  
  \end{aligned}
\end{equation*}

因此

\begin{equation}
  \label{eq:a1141}
  \begin{aligned}
    \vec{\Jmath}_f = \sigma \vec{E} = \dfrac{\sigma \lambda_f}{2 \pi \varepsilon r} \vec{e}_r 
    \quad\quad
    \vec{\Jmath}_D = \dfrac{\partial \vec{D}}{\partial t} = \dfrac{1}{2 \pi r} \dfrac{\partial \lambda_f}{\partial t} \vec{e}_r  
  \end{aligned}
\end{equation}

又因为

\begin{equation*}
  \begin{aligned}
    \nabla \cdot \vec{D} = \rho_f
    \quad\quad
    \nabla \cdot \vec{\Jmath}_f + \dfrac{\partial \lambda_f}{\partial t} = 0 
  \end{aligned}
\end{equation*}

即

\begin{equation*}
  \begin{aligned}
    - \dfrac{\lambda_f}{2 \pi r^2} = \rho_f
    \Rightarrow
    \dfrac{\partial \rho_f}{\partial t} = - \dfrac{1}{2 \pi r^2} \dfrac{\partial \lambda_f}{\partial t}   
  \end{aligned}
\end{equation*}

\begin{equation*}
  \begin{aligned}
    \nabla \cdot \vec{\Jmath}_f = - \dfrac{\sigma \lambda_f}{2 \pi \varepsilon r^2} 
  \end{aligned}
\end{equation*}

所以

\begin{equation*}
  \begin{aligned}
    - \dfrac{\sigma \lambda_f}{2 \pi \varepsilon r^2} - \dfrac{1}{2 \pi r^2} \dfrac{\partial \lambda_f}{\partial t} = 0 
  \end{aligned}
\end{equation*}

即

\begin{equation*}
  \begin{aligned}
    \dfrac{\partial \lambda_f}{\partial t} = - \dfrac{\sigma}{\varepsilon} \lambda_f  
  \end{aligned}
\end{equation*}

代入式\ref{eq:a1141}可得

\begin{equation*}
  \begin{aligned}
    \vec{\Jmath}_D + \vec{\Jmath}_f = 0
  \end{aligned}
\end{equation*}

传导电流和位移电流完全抵消。

因为

\begin{equation*}
  \begin{aligned}
    \nabla \times \vec{H} = \vec{\Jmath}_f + \dfrac{\partial \vec{D}}{\partial t} 
  \end{aligned}
\end{equation*}

所以电容器内无磁场。

$\lambda_f$衰减的规律为:

\begin{equation*}
  \begin{aligned}
    \lambda_f \left( t \right) = \lambda_f \left( 0 \right) \exp \left[ - \dfrac{\sigma}{\varepsilon} t  \right]
  \end{aligned}
\end{equation*}

介质中能量耗散的功率密度为

\begin{equation*}
  \begin{aligned}
    p = \vec{\Jmath}_f \cdot \vec{E} = \sigma E^2 = \sigma \left( \dfrac{\lambda_f}{2 \pi \varepsilon r}  \right)^2
  \end{aligned}
\end{equation*}

长为$l$的介质耗散功率为

\begin{equation*}
  \begin{aligned}
    \int_a^b 2 \pi r l p \md r = \dfrac{\sigma l \lambda_f^2}{2 \pi \varepsilon^2} \ln \dfrac{b}{a}  
  \end{aligned}
\end{equation*}

电场能量密度变化率为

\begin{equation*}
  \begin{aligned}
    \dfrac{\partial}{\partial t } \left[ \dfrac{1}{2} \vec{E} \cdot \vec{D}  \right] = \dfrac{\partial}{\partial t} \left[ \dfrac{1}{2} \dfrac{\lambda_f^2}{\varepsilon \left( 2 \pi r \right)^2}   \right]  = - \sigma \left( \dfrac{\lambda_f}{2 \pi \varepsilon r}  \right)^2
  \end{aligned}
\end{equation*}

长为$l$的介质电场能量变化率为

\begin{equation*}
  \begin{aligned}
    \int_a^b - \sigma \left( \dfrac{\lambda_f}{2 \pi \varepsilon r}  \right)^2 \md r = - \dfrac{\sigma l \lambda_f^2}{2 \pi \varepsilon^2} \ln \dfrac{b}{a}  
  \end{aligned}
\end{equation*}

介质耗散的能量恰好等于电场减少的能量

\end{document} 
