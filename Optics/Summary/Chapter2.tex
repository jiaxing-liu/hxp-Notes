\chapter{The Propagation of Light}

\section{Scattering and Interference}

Destructive interference of the scattering light

\begin{itemize}
\item The denser the substance through which light advances, the less the lateral scattering.
\item The longer the wavelength, the less the lateral scattering.
\item On an overcast day, sky looks white because of large water droplets scatters all lights. On sunny day, sky only scatters blue light. And if there were no atmosphere, sky would be black as it is on moon.
\item All molecules have electronic resonances in UV, the closer driving frequency is to a resonance, the more vigorously the oscillator responds. Blue and violet response more than red, sky is blue.
\end{itemize}

\section{Speed of Light in Medium}

\begin{figure}[H]
  \centering
  \begin{subfigure}{.45\textwidth}
    \centering
    \includegraphics[width=\linewidth]{figures/Light-speed-in-dielectric1.png}
  \end{subfigure}
  \begin{subfigure}{.45\textwidth}
    \centering
    \includegraphics[width=\linewidth]{figures/Light-speed-in-dielectric2.png}
  \end{subfigure}
\end{figure}

If the phase of light in dielectric lags behind vacuum one, the resultant lags, and vice versa.

\section{Internal and External Reflection}

\begin{figure}[H]
  \centering
  \includegraphics[width=0.8\linewidth]{figures/internal-external-reflection}
\end{figure}

Beam I (internal reflection) and Beam II (external reflection) has $180^{\circ}$ phase shift, when the gap between right part and left part of the glass in picture b becomes zero, two beams diminishes. This case is the same as picture a where the glass has not been cut.

\section{The Fresnel Equations}

\subsection{Electric Field Perpendicular to Plane of Incidence}

\begin{figure}[H]
  \centering
  \includegraphics[width=0.4\linewidth]{figures/Fresnel-perpendicular}
\end{figure}

\begin{equation*}
 \left\{
  \begin{aligned}
    & E_i + E_r = E_t \\
    & B_i \cos \theta_i = B_r \cos \theta_r + B_t \cos \theta_t
  \end{aligned}
  \right.
  \quad \Rightarrow \quad
  \left\{
  \begin{aligned}
    & E = v B \\
    & v = \dfrac{c}{n}
  \end{aligned}
  \right.
\end{equation*}

We define the amplitude reflection coefficient $r$, the amplitude transmission coefficient $t$

\begin{equation*}
  \left\{
  \begin{aligned}
    r = \dfrac{n_i \cos \theta_i - n_t \cos \theta_t}{n_i \cos \theta_i + n_t \cos \theta_t} \\
    t = \dfrac{2 n_i \cos \theta_i}{n_i \cos \theta_i + n_t \cos \theta_t} 
  \end{aligned}
  \right.
  \quad + \quad
  \begin{aligned}
    n_i \sin \theta_i = n_t \sin \theta_t
  \end{aligned}
  \quad \Rightarrow \quad
  \left\{
  \begin{aligned}
    r_{\perp} &= \dfrac{\sin \left( \theta_i - \theta_t \right)}{\sin \left( \theta_i + \theta_t \right)} \\
    t_{\perp} &= \dfrac{2 \sin \theta_t \cos \theta_i}{\sin \left( \theta_i + \theta_t \right)} 
  \end{aligned}
  \right.
\end{equation*}

\subsection{Electric Field Parallel to Plane of Incidence}

\begin{figure}[H]
  \centering
  \includegraphics[width=0.4\linewidth]{figures/Fresnel-parallel}
\end{figure}

\begin{equation*}
 \left\{
  \begin{aligned}
    & B_i + B_r = B_t \\
    & E_i \cos \theta_i = E_r \cos \theta_r + E_t \cos \theta_t
  \end{aligned}
  \right.
  \quad \Rightarrow \quad
  \left\{
  \begin{aligned}
    & E = v B \\
    & v = \dfrac{c}{n}
  \end{aligned}
  \right.
\end{equation*}

We define the amplitude reflection coefficient $r$, the amplitude transmission coefficient $t$

\begin{equation*}
  \left\{
  \begin{aligned}
    r_{\parallel} = \dfrac{n_t \cos \theta_i - n_i \cos \theta_t}{n_t \cos \theta_i + n_i \cos \theta_t} \\
    t_{\parallel} = \dfrac{2 n_i \cos \theta_i}{n_t \cos \theta_i + n_i \cos \theta_t} 
  \end{aligned}
  \right.
  \quad + \quad
  \begin{aligned}
    n_i \sin \theta_i = n_t \sin \theta_t
  \end{aligned}
  \quad \Rightarrow \quad
  \left\{
  \begin{aligned}
    r_{\parallel} &= \dfrac{\sin \left( 2 \theta_i \right) - \sin \left( 2 \theta_t \right)}{\sin \left( 2 \theta_i \right) + \sin \left( 2 \theta_t \right)} = \dfrac{\tan \left( \theta_i - \theta_t \right)}{\tan \left( \theta_i + \theta_t \right)} \\
    t_{\parallel} &= \dfrac{2 \sin \theta_t \theta_i}{\sin \left( \theta_i + \theta_t \right) \cos \left( \theta_i - \theta_t \right)} 
  \end{aligned}
  \right.
\end{equation*}

\section{Polarization Angle}

\begin{equation*}
  \begin{aligned}
    \theta_p = \theta_i = \dfrac{\pi}{2} - \theta_t \quad + \quad  n_i \sin \theta_i = n_t \sin \theta_t \quad \Rightarrow \quad \theta_p = \arctan \dfrac{n_t}{n_i} 
  \end{aligned}
\end{equation*}

\begin{figure}[H]
  \centering
  \includegraphics[width=0.5\linewidth]{figures/Polarization-angle}
\end{figure}

\section{Critical Angle}

\begin{equation*}
  \begin{aligned}
    \theta_c = \arcsin \left( \dfrac{n_t}{n_i}  \right)
  \end{aligned}
\end{equation*}

\section{Phase Shift}

\subsection{Outer reflection}

While $n_i < n_t$ (Outer reflection)

$\theta_i = 0$

\begin{equation*}
  \begin{aligned}
    r_{\perp} &= - r_{\parallel} = \dfrac{n_i - n_t}{n_i + n_t} \\
    t_{\parallel} &= \phantom{+} t_{\perp} = \dfrac{2 n_i}{n_i + n_t} 
  \end{aligned}
  \quad \Rightarrow \quad 
  \left\{
  \begin{aligned}
    & r_{\perp} + t_{\parallel} = 1 \\
    & r_{\parallel} > 0 \quad\quad \left( \text{inverted} \right) \\
    & r_{\perp} < 0 \quad\quad \left( \text{inverted} \right)
  \end{aligned}
  \right.
\end{equation*}

$\theta_i = \dfrac{\pi}{2} $

\begin{equation*}
  \begin{aligned}
    r_{\perp} = -1 \quad \left( \text{inverted} \right) \quad\quad r_{\parallel} = -1 \quad \left( \text{inverted} \right) \quad\quad t_{\perp} = 0 \quad\quad t_{\parallel} = 0
  \end{aligned}
\end{equation*}

Outer reflection $\quad \Rightarrow \quad $ phase shifted by $\pi$.

\subsection{Inner reflection}

$\theta_i = 0$

\begin{equation*}
  \begin{aligned}
    r_{\perp} &= - r_{\parallel} = \dfrac{n_i - n_t}{n_i + n_t} \\
    t_{\parallel} &= \phantom{+} t_{\perp} = \dfrac{2 n_i}{n_i + n_t} 
  \end{aligned}
  \quad \Rightarrow \quad 
  \left\{
  \begin{aligned}
    & r_{\perp} + t_{\parallel} = 1 \\
    & r_{\parallel} < 0 \quad\quad \left( \text{non-inverted} \right) \\
    & r_{\perp} > 0 \quad\quad \left( \text{non-inverted} \right)
  \end{aligned}
  \right.
\end{equation*}

Inner reflection $\quad \Rightarrow \quad $ phase shifted by $0$.

\section{Reflectance and Transmittance}

\begin{figure}[H]
  \centering
  \includegraphics[width=0.4\linewidth]{figures/Reflectance-and-Transmittance}
\end{figure}

\begin{equation*}
  \left\{
  \begin{aligned}
    R &= \dfrac{I_r A \cos \theta_r}{I_i A \cos \theta_i} = \dfrac{I_r}{I_i} \\
    T &= \dfrac{I_t A \cos \theta_t}{I_i A \cos \theta_i} = \dfrac{I_t \cos \theta_t}{I_i \cos \theta_i}
  \end{aligned}
  \right.
\end{equation*}

\begin{equation*}
  \begin{aligned}
    \vec{S} = c^2 \varepsilon_0 \vec{E} \times \vec{B}
  \end{aligned}
  \quad \Rightarrow \quad
  \begin{aligned}
    I = \dfrac{1}{2} \varepsilon v E_0^2 = \dfrac{1}{2} \varepsilon_0 \varepsilon_r v E_0^2 = \dfrac{1}{2} \varepsilon_0 n^2 v E_0^2 = \dfrac{1}{2} \varepsilon_0 n c E_0^2
  \end{aligned}
\end{equation*}

\begin{equation*}
  \left\{
  \begin{aligned}
    R &= \dfrac{I_r}{I_i} = \left( \dfrac{E_{0r}}{E_{0i}}  \right)^2 = r^2 \\
    T &= \dfrac{I_t \cos \theta_t}{I_i \cos \theta_i} = \left( \dfrac{n_t \cos \theta_t}{n_i \cos \theta_i}  \right) \left( \dfrac{E_{0t}}{E_{0i}}  \right)^2 = \left( \dfrac{n_t \cos \theta_t}{n_i \cos \theta_i}  \right) t^2
  \end{aligned}
  \right.
  \quad \Rightarrow \quad
  \left\{
  \begin{aligned}
    & R_{\perp} = r_{\perp}^2 \\
    & R_{\parallel} = r_{\parallel}^2 \\
    & T_{\perp} = \left( \dfrac{n_t \cos \theta_t}{n_i \cos \theta_i}  \right) t_{\perp}^2 \\
    & T_{\parallel} = \left( \dfrac{n_t \cos \theta_t}{n_i \cos \theta_i}  \right) t_{\parallel}^2
  \end{aligned}
  \right.
  \quad \Rightarrow \quad
  \left\{
  \begin{aligned}
    & R_{\perp} + T_{\perp} = 1 \\
    & R_{\parallel} + T_{\parallel} = 1 \\
    & R + T = 1 \\
  \end{aligned}
  \right.
\end{equation*}

When $\theta_i = 0$, any distinction between the parallel and perpendicular components of R and T vanishes. Thus

\begin{equation*}
  \left\{
  \begin{aligned}
    & R = R_{\parallel} = R_{\perp} = \left( \dfrac{n_t - n_i}{n_t + n_i}  \right)^2 \\
    & T = T_{\parallel} = T_{\perp} = \dfrac{4 n_t n_i}{\left( n_i + n_t \right)^2} 
  \end{aligned}
  \right.
\end{equation*}

\section{The Evanescent Wave}

\begin{figure}[H]
  \centering
  \includegraphics[width=0.6\linewidth]{figures/Evanescent-wave}
\end{figure}

\begin{equation*}
  \begin{aligned}
    \vec{E}_t = \vec{E}_{0t} \exp \left[ i \left( \vec{k}_t  \cdot \vec{r} - \omega t \right) \right]
  \end{aligned}
  \quad\quad 
  \begin{aligned}
    \vec{k}_t \cdot \vec{r} = k_{tx} x + k_{ty} y
  \end{aligned}
  \quad\quad 
  \begin{aligned}
    k_t = n_t k_0 = \dfrac{2\pi}{\lambda} 
  \end{aligned}
\end{equation*}

\begin{equation*}
  \left\{
  \begin{aligned}
    k_{tx} &= k_t \sin \theta_t = \left( \dfrac{n_i}{n_t}  \right) k_t \sin \theta_i = n_i k_0 \sin \theta_i \\
    k_{ty} &= k_t \cos \theta_t = i k_t \sqrt{\dfrac{n_i^2 \sin^2 \theta_i}{n_t^2} - 1} = i \beta
  \end{aligned}
  \right.
  \quad \Rightarrow \quad 
  \begin{aligned}
    \vec{E}_t = \vec{E}_{0t} \exp \left( - \beta y \right) \exp \left[ i \left( n_i k_0 x \sin \theta_i - \omega t \right) \right]
  \end{aligned}
\end{equation*}

\section{Optical Properties of Metals}

The index of refraction of metal is complex

\begin{equation*}
  \begin{aligned}
    \tilde{n} = n_R - i n_I
  \end{aligned}
\end{equation*}

\begin{equation*}
  \begin{aligned}
    \nabla \times \vec{H} = \varepsilon_0 \varepsilon_r \dfrac{\partial \vec{E}}{\partial t} + \sigma \vec{E} = - i \omega \varepsilon_0 \varepsilon_r \vec{E} + \sigma \vec{E} = - i \omega \varepsilon_0 \tilde{\varepsilon}_r \vec{E}
  \end{aligned}
\end{equation*}

Whereas

\begin{equation*}
  \begin{aligned}
    \tilde{\varepsilon}_r = \varepsilon_r + i \dfrac{\sigma}{\omega \varepsilon_0} 
  \end{aligned}
\end{equation*}

\begin{equation*}
  \begin{aligned}
    \tilde{n}^2 = \tilde{\varepsilon}_r = \varepsilon_r + i \dfrac{\sigma}{\omega \varepsilon_0} = \left( n_R + i n_I \right)^2
  \end{aligned}
\end{equation*}

Since $\dfrac{\sigma}{\omega \varepsilon_0 \varepsilon_r} \gg 1 $

\begin{equation*}
  \begin{aligned}
    n_I \approx n_R = \sqrt{\dfrac{\sigma}{2 \omega \varepsilon_0} }
  \end{aligned}
\end{equation*}

Skin depth

\begin{equation*}
  \begin{aligned}
    \delta = \sqrt{\dfrac{1}{2 \omega \mu_0 \sigma} }
  \end{aligned}
\end{equation*}

Reflectance

\begin{equation*}
  \begin{aligned}
    R = \left| \dfrac{n_i - n_t}{n_i + n_t}  \right|^2 = \left( \dfrac{\tilde{n} - 1}{\tilde{n} + 1}  \right) \left( \dfrac{\tilde{n} - 1}{\tilde{n} + 1}  \right)^{*} = \dfrac{\left( n_R - 1 \right)^2 + n_I^2}{\left( n_R + 1 \right)^2 + n_I^2} 
  \end{aligned}
\end{equation*}

%%% Local Variables:
%%% mode: latex
%%% TeX-master: "Optics"
%%% End:
