\documentclass{article}
% Chinese
% \documentclass[UTF8, nofonts, mathptmx, 12pt, onecolumn]{article}
% \usepackage{xeCJK}
% \setCJKmainfont{SimSun}
\usepackage{amsmath}
\usepackage{amsfonts}
\usepackage{amssymb}
\usepackage{wasysym}
%\usepackage{ctex}
\usepackage{graphicx}
\usepackage{float}
\usepackage{geometry}
\geometry{a4paper,scale=0.8}
\usepackage{caption}
\usepackage{subcaption}
% \newcommand{\oiint}{\mathop{{\int\!\!\!\!\!\int}\mkern-21mu \bigcirc} {}}
\newcommand*{\dif}{\mathop{}\!\mathrm{d}}
\newcommand*{\md}{\mathop{}\!\mathrm{d}}
\newcommand*{\me}{\mathrm{e}}

\usepackage{parskip}
\setlength{\parindent}{0cm}

\usepackage{bm}
\let\Oldmathbf\mathbf
\renewcommand{\mathbf}[1]{\boldsymbol{\Oldmathbf{#1}}}
\let\eqnarray\align
\usepackage{multicol}
\usepackage{authblk}
\author{Xiping Hu}
\affil{http://thehxp.tech/}
\title{Homework for Chapter I}

\begin{document}
\maketitle
 \section{Why does lights propagates in a straight line? In which way the direction of propagation of light may alter?}

Two natural phenomena that occur due to the characteristic of light travelling in a straight line are :
\begin{itemize}
\item Shadows of objects just opposite to the source.
\item A lunar eclipse that occurs when the Earth is situated between the Moon and the Sun. The shadow of the Earth falls on the surface of the Moon to appear invisible.
\end{itemize}

Ways to alter the direction light travels:

\begin{itemize}
\item Use prisms to redirect the light to a different directions, for example, a triangle prism which has two 45 degree angles can cause the light change its direction by 90 degree.
\item Mirrors may reflect the lights.
\item Rayleigh scattering
  
\end{itemize}

\section{How are the frequency of light and it's color related? Investigate the mechanism of our eyes distinguishing different colors.}

The color of light is related with electro-magnetic spectrum. Different wavelength of light has different colors. For instance, for light whose wavelength is between 700–635 nm, the color is red. And the green colored light's wavelength is between 560–520 nm.

Our eyes has several cells that can distinguish different wavelength of lights. The ability of the human eye to distinguish colors is based upon the varying sensitivity of different cells in the retina to light of different wavelengths.

\section{What is the main difference between Newton's particle theory of light and the quantum theory of light?}

In Newton's Theory of light, lights, are treated as extremely small particles.

Quantum theory tells us that both light and matter consists of tiny particles which have wavelike properties associated with them. Light is composed of particles called photons, and matter is composed of particles called electrons, protons, neutrons. It's only when the mass of a particle gets small enough that its wavelike properties show up.

Quantum theory of light is the combination of the particle and the wave theory of light. And it can explain some phenomenon like blackbody radiation, well particle theory can not.

\section{List some phenomenon associated with the wave nature of light.}

\begin{itemize}
\item The polarization of light.
\item The interference of light.
\item The diffraction of light.
\item The addition of light: Two different waves of light may cancel if they are of the same frequency, and one is one-half out-of-phase with the other.
  
\end{itemize}

\section{What is the wave-particle duality of light? Show some phenomenon to illustrate the wave or particle nature of light.}

The wave-particle theory tells us that light consists of tiny particles, which have wavelike properties. Unless we discuss light in microcosmic situations, the wavelike properties can be omitted.

For example, the interference of light may illustrate the wave nature of light. And the reflection law can be explained by the particle nature of light.

\section{What are the basic attributes of light? Point out whether it represents the wave or the particle nature of light.}

The basic attributes of light are:

\begin{itemize}
\item Light travels in a straight line.
\item Reflection
\item Radiation
  
\end{itemize}

While the first and seconds attribute above represents the particle nature, the third attribute represents the wave nature.

\section{What are the main stages of the history of optics? List the representative figures and their discovery in each stage.}

From the 17th century

\begin{itemize}
\item Johannes Kepler discovered total reflection.
\item Willebrord Snell discovered law of reflection.
\item Pierre de Fermat rederived law of reflection from his own Principle of Least Time.
\item Professor Francesco Maria Grimaldi noted the phenomenon of refraction.
\item Newton concluded that white light was composed of a mixture of a whole independent colors.
\item Huygens correctly concluded that light effectively slowed down on entering more dense media.
\end{itemize}

In the 19th century
\begin{itemize}
\item The wave theory was reborn at the hands of Dr. Thomas Young.
\item Louis Fizeau determined the speed of light.
\item Michael Faraday established an interrelationship between electromagnetism and light.
\item Maxwell showed that the electromagnetic field could propagate as a transverse wave theoretically.
\item Hertz verified the existence of long electromagnet wave.
\end{itemize}

20th century optics
\begin{itemize}
\item Einstein introduced his Special Theory of Relativity.
  \item Bohr, Born, Heisenburg, Schrodinger, De Broglie, Pauli, Dirac, Plank and others established quantum mechanics.
\end{itemize}

\section{Discuss: How do you think the optical technology will affect human society in the future?}

\begin{itemize}
\item Some advanced optical technology such as laser cooling, may lead to revolutionary discoveries.
\item Laser technologies may help the development of industry technologies.
\end{itemize}

\end{document}